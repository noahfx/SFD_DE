\documentclass{beamer}
\usepackage[utf8]{inputenc}
\usepackage[spanish]{babel}
\usepackage{hyperref}
\usepackage{ulem}\normalem
\usepackage{graphicx}
% \input                                                                                                                                                                  
\parskip 10.9pt
\usetheme{Darmstadt}
\usecolortheme {beaver}
\title{Ambientes de Escritorio}
\author[noahfx]{Josu\'e Ortega  \\ \texttt{josueortega.org}}
\institute{LUGUSAC}
\begin{document}
\begin{frame}

  \titlepage
   \begin{figure}[htb]
  \includegraphics[width=0.1\textwidth]{img/sfd_logo.png}
  \end{figure}
    {\tiny
    \begin{center}
      \begin{tabular}{l@{\hspace{1em}}l}
         licencia
        & \href{http://creativecommons.org/licenses/by-sa/3.0/}{CC BY-SA 3.0 ---
          Creative Commons Attribution-ShareAlike 3.0} \\
      \end{tabular}
    \end{center}}
\end{frame}
\section {¿Qu\'e es un ambiente de escritorio?}
\begin{frame}{¿Qu\'e es un ambiente de escritorio?}
\begin{itemize} 
\item Conjunto de software para ofrecer al usuario una interaccion amigable y comoda. 
\item Interfaz Grafica que ofrece facilidades de acceso y configuracion.
\end{itemize}
\end{frame}
\begin{frame}
  \begin{figure}[htb]
  \includegraphics[width=0.7\textwidth]{img/des.jpg}
  \end{figure}
\end{frame}

\section{KDE}
\begin{frame}{KDE}
 \begin{itemize}
 \item {\bf Sitio web:} http://kde.org
 \item K Desktop Enviroment
 \item Qt
 \end{itemize}
\pause
\alert {Pros:}
\begin{itemize}
  \item F\'acil de personalizar
  \item F\'acil de entender y dominarlo si se esta acostumbrado interfaces
    con menus.
  \item KDE contiene una gran variedad de aplicaciones
\end{itemize}
\alert {Contras:}
\begin{itemize}
  \item Se necesitan muchos recursos para echarlo a andar. 
\end{itemize}


\end{frame}

\section{GNOME}
\begin{frame}{GNOME}
 \begin{itemize}
 \item {\bf Sitio web:} http://www.gnome.org
 \item Proyecto iniciado por un par de mexicanos
 \item Fue desarrollado como una alternativa a KDE
 \item El cambio de Gnome2 a Gnome3 caus\'o mucha controversia
 \item GTK
 \end{itemize}
\pause
\alert {Pros:}
\begin{itemize}
  \item F\'acil de usar
  \item Integraci\'on con cuentas online y de correo electronico
  \item El menu soporta agregar favoritos, para rapido acceso a programas mas usados.
\end{itemize}
\end{frame}
\begin{frame}
\alert {Contras:}
\begin{itemize}
  \item Pocas opciones de personalizacion
  \item De cierta manera te obliga a usar la computadora a como Gnome dice.
  \item Puede no funcionar bien en hardware un poco viejo. 
\end{itemize}
\end{frame}

\section{XFCE}
\begin{frame}{XFCE}
 \begin{itemize}
 \item {\bf Sitio web:} http://www.xfce.org
 \item Esta dise\~nado para ser ligero y rapido sin dejar de ser {\em bonito}
 \item Fue desarrollado 1996
 \end{itemize}
\pause
\alert {Pros:}
\begin{itemize}
  \item Bastante ligero, corre en computadoras viejas.
  \item Provee un conjunto decente de features si tanto bloat.
  \item Altamente personalizable
\end{itemize}
\end{frame}
\section {Otros Entornos}
\begin{frame}{Otros Entornos}
\begin{itemize}
\item LXDE
\item Mate
\item Unity
\item Cinnamon
\end{itemize}
\end{frame}

\section {¿Preguntas?}
\begin{frame}
      \huge ¿Preguntas? \\
      \pause
       \alert { Gracias :)} \\
     
      
       
         {\tiny
    \begin{center}
       \href{mailto:josueortega@debian.org.gt}{josueortega@debian.org.gt}\\
      \begin{tabular}{l@{\hspace{1em}}l}
       
    
        
       licencia
        & \href{http://creativecommons.org/licenses/by-sa/3.0/}{CC BY-SA 3.0 ---
          Creative Commons Attribution-ShareAlike 3.0} \\
       
      \end{tabular}
    \end{center}}
\end{frame}




\end{document} 

